\section{總結}
  本論文探討了結構化支撐向量機被拿來做語音辨識的可能性。 
  雖然目前的模型還無法做小字彙或大字彙語音辨識,
  但在音素辨識上已經能取得不錯的部份成果。
  音素正確率在使用音素事後機率作為輸入特徵的情況下能贏過隱藏式馬可夫模型絕對正確率的1\%。

  結構化支撐向量機帶來好的表現正如支撐向量機常常帶來不錯的表現。
  不過同時,
  它也繼承了支撐向量機的一項缺點,
  就是速度上的問題。
  雖然支撐向量機不是需要非常長訓練時間的機器學習模型,
  但比起其他像是隱藏式馬可夫模型等一類簡單的貝氏學派模型。
  支撐向量機碰到訓練集非常大的時候會難以應付。
  而偏偏在語音辨識上,
  特別在聲學模型的訓練上,
  訓練集往往是非常大的
  (上百個小時的資料再切成音框)。
  正因為這一點,
  結構化支撐向量機在語音辨識的應用上還有很大的問題要克服
  (當然也要先克服小字彙、大字彙辨識的問題)。

\section{未來展望}
  如總結一節所說的,
  結構化支撐向量機模型還未考慮小字彙辨識或大字彙辨識、
  以及訓練速度上的問題。
  其實同樣的問題也出現在類似的鑑別式模型,
  像是條件隨機域(Conditional Random Field; CRF)。
  相信只要其中一種鑑別式模型解決了如何做小字彙或大字彙辨識的問題。
  其他鑑別式模型也能尋同樣的模式解決,
  開闢出除了鑑別式訓練法則(Discrminative Training)外其他一條路徑
  (雖然鑑別式訓練法則其實就是一種鑑別式模型)。
  相信到時候很多領域類似的問題也能同時被解決。

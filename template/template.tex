%
% this file is encoded in utf-8
% v2.0 (Apr. 5, 2009)

\documentclass[12pt, a4paper]{ntuthesis}

% 除非校方修改了論文格式 (margins, header, footer, 浮水印, 中文數字之章別)
% 或者需要增加所用的 LaTeX 套件,
% 或者要改預設中文字型、編碼
% 否則毋須修改本檔內容
% 論文撰寫,請修改以 my_  開頭檔名的各檔案

\usepackage{CJKutf8}  %%% ZZZ %%% macro for Chinese/Japanese/Korean processing
\usepackage{CJKnumb} %%% ZZZ %%% for Chinese numbering capability
\usepackage[nospace]{cite}  % for smart citation
%\usepackage{geometry}  % for easy margin settings
\usepackage{ntuthesis}

%
% margins setting
%\geometry{verbose,a4paper,tmargin=3.5cm,bmargin=2cm,lmargin=4cm,rmargin=2cm}
%

% 插圖套件 graphicx
% 使用者工作流程是用 pdftex 還是 latex + dvipdfmx?
% 視情況而有不同的參數
% 這裡作自動判斷
% 參考自
% http://www.tex.ac.uk/cgi-bin/texfaq2html?label=ifpdf
\newcommand\mydvipdfmxflow{dvipdfmx}
\newcommand\mypdftexflow{pdftex}
\ifx\pdfoutput\undefined
  % not running pdftex
  \usepackage[dvipdfm]{graphicx}
  \newcommand\myworkflow{dvipdfmx}  % set the flag for hyperref
\else
  \ifx\pdfoutput\relax
    % not running pdftex
    \usepackage[dvipdfm]{graphicx}
    \newcommand\myworkflow{dvipdfmx}  % set the flag
  \else
    % running pdftex, with...
    \ifnum\pdfoutput>0
      % ... PDF output
      \usepackage[pdftex]{graphicx}
      \newcommand\myworkflow{pdftex}  % set the flag
    \else
      %...DVI output
      \usepackage[dvipdfm]{graphicx}
      \newcommand\myworkflow{dvipdfmx}  % set the flag
    \fi
  \fi
\fi

% 增強功能型頁楣 / 頁腳套件
\usepackage{fancyhdr}  % 借用此套件來擺放浮水印 
% (佔用了 central header)
% 不需要浮水印的使用者仍可利用此套件,產生所需的 header, footer
%
% 啟動 fancy header/footer 套件
\pagestyle{fancy}
\fancyhead{}  % reset left, central, right header to empty
\fancyfoot[C]{\thepage} %中間 footer 擺放頁碼
\renewcommand{\headrulewidth}{0pt} % header 的直線; 0pt 則無線

% 如果不需要任何浮水印,則請把下列介於 >>> 與 <<< 之間
% 的文字行關掉 (行首加上百分號)
%% 浮水印 >>> 
%\input{yzu_watermark.tex}
%% <<< 浮水印

% 如需額外的頁楣 (header) 或 footer,請在 my_headerfooter.tex 裡依例修改
% 它的預設內容是都關掉,可依需要打開
%
% this file is encoded in utf-8
% v2.0 (Apr. 5, 2009)

%%%%%%% 其他的 header (left, right) 定義
% 底下定義了一些常見的 header 型式
% 預設情況是關掉的
% 使用者可以視需要將之打開
% 也就是把下列介於 >>> 與 <<< 之間
% 的文字行打開 (行首去掉百分號)

%% header >>>
%\renewcommand{\chaptermark}[1]{%
%\markboth{\prechaptername\ \thechapter\ \postchaptername%
%\ #1}{}%
%}  %定義 header 使用的「章」層級的戳記
%\fancyhead[L]{} % 左 header 為空
%\fancyhead[R]{\leftmark}  % 右 header 擺放「章」層級的戳記 (以 \leftmark 叫出)
%\renewcommand{\headrulewidth}{0.4pt}  % header 的直線 0.4pt; 0pt 則無線
%% <<< header

%%%%%%% 其他的 footer (left, right) 定義
% 底下定義了一些常見的 footer 型式
% 預設情況是關掉的
% 使用者可以視需要將之打開
% 也就是把下列介於 >>> 與 <<< 之間
% 的文字行打開 (行首去掉百分號)

%% footer >>>
%\fancyfoot[L]{} % 左 footer 為空
%\fancyfoot[R]{\small{YZU \LaTeX\ v2.0}} % 右 footer 擺放論文格式版本
%\renewcommand{\footrulewidth}{0.4 pt} % footer 的直線 0.4pt; 0pt 則無線
%% <<< footer




%%%%%%%%%%%%%%%%%%%%%%%%%%%%%%
%%%% 非必要的套件,但很實用
\usepackage{amsmath} % 各式 AMS 數學功能
\usepackage{amssymb} % 各式 AMS 數學符號
\usepackage{mathrsfs} %草寫體數學符號,在數學模式裡用 \mathscr{E} 得草寫 E
\usepackage{listings} % 程式列表套件
\usepackage{subfig}
\usepackage{tabularx}
\usepackage{url}
\usepackage[usenames,dvipsnames]{xcolor}
\usepackage{pgf}
\usepackage{tikz}
\usetikzlibrary{arrows,automata,positioning}



% Title Page
\renewcommand{\enTitle}{English Title}  %英文標題
\renewcommand{\zhTitle}{中文標題}  %中文標題
\renewcommand{\authorZhName}{王小明}  %作者中文姓名
\renewcommand{\authorEnName}{Wang, Shiao-Ming}  %作者英文姓名
\renewcommand{\authorStudentID}{R99999999}  %作者學號
\renewcommand{\advisorZhName}{吳大明}  %指導教授中文姓名
\renewcommand{\advisorEnName}{Wu, Da-Ming}  %指導教授英文姓名
\renewcommand{\zhCollegeName}{電機資訊學院}  %學院中文名稱
\renewcommand{\enCollegeName}{College of Electrical Enginnering and Computer Science}  %學院英文名稱
\renewcommand{\zhDepartmentName}{Latex研究所}  %系所中文名稱
\renewcommand{\enDepartmentName}{Graduate Institute of Computer Science and Information Engineering}  %系所英文名稱
\renewcommand{\rocYear}{九十八}  %中華民國紀年年份
\renewcommand{\zhMonth}{六}  %中文月份
\renewcommand{\enYear}{2009}  %公元紀年
\renewcommand{\enMonth}{January}  %英文月份
\renewcommand{\oralDate}{97 年 1 月 1 日}  %口試日期

%
% listing setting
\lstset{breaklines=true,% 過長的程式行可斷行
extendedchars=false,% 中文處理不需要 extendedchars
texcl=true,% 中文註解需要有 TeX 處理過的 comment line, 所以設成 true
comment=[l]\%\%,% 以雙「百分號」做為程式中文註解的起頭標記,配合 MATLAB
basicstyle=\small,% 小號字體, 約 10 pt 大小
commentstyle=\upshape,% 預設是斜體字,會影響註解裏的英文,改用正體
%language=Octave % 會將一些 octave 指令以粗體顯示
}

\usepackage{url} % 在文稿中引用網址,可以用 \url{http://www.yzu.edu.tw} 方式

%%%% 以上為非必要套件
%%%%%%%%%%%%%%%%%%%%%%%%%%%%%%

%%% 以下是 hyperref 套件
%%%%%%%%%%%%%%%%%%%%%%%%%%%%%%
% hyperref 會擾亂 cite.sty 對文獻號碼縮編的排版,所以依據
% http://www.ctan.org/tex-archive/macros/latex/contrib/hyperref/
% 作如下的更動,使得 hyperref 不做文獻號碼的超連結。
\makeatletter
\def\NAT@parse{\typeout{This is a fake Natbib command to fool Hyperref.}}
\makeatother

% hyperlinkable table of contents
% 章節目錄、圖表超連結
\ifx\myworkflow\mydvipdfmxflow
	\usepackage[dvipdfmx, debug, colorlinks, linkcolor=black, citecolor=black, urlcolor=black, unicode]{hyperref}
\else
	\usepackage[pdftex, debug, colorlinks, linkcolor=black, citecolor=black, urlcolor=black, unicode]{hyperref}	
\fi

% if hyperref is not used (e.g., in LyX application)
% define dummy \phantomsection for those occurences
%   in yzu_frontpages.tex, yzu_backpages.tex, my_appendix.tex
\ifx\hypersetup\undefined
	\newcommand\phantomsection{}
\fi

% hyperref跟algorithm衝突,hyperref必須放在algorithm前面
\usepackage{algorithm}
\usepackage{algorithmic}
%%%% 以上為所有套件
%%%% 
%%%% 

% global page layout
%\newcommand{\mybaselinestretch}{1.5}  %行距 1.5 倍 + 20%, (約為 double space)
%\renewcommand{\baselinestretch}{\mybaselinestretch}  % 論文行距預設值
%\parskip=2ex  % 段落之間的間隔為兩個 x 的高度
%\parindent = 0Pt  % 段首內縮由 CJK 控制,所以這裡就設成不內縮

%%%%%%%%%%%%%%%%%%%%%%%%%%%%%
%  end of preamble
%%%%%%%%%%%%%%%%%%%%%%%%%%%%%
%
\begin{document}
\begin{CJK}{UTF8}{bsmi}   %%% ZZZ %%%  <<< 在這裡更改預設中文字型、編碼
% 編碼:UTF8, Bg5, ...
% 中文字型名稱:TeXLive 安裝有一套明體字 bsmi, 楷書與其他字型視你的 LaTeX CJK 系統裝設情況而定

% 針對 latex + dvipdfmx 工作流程在 hyperref 套件的影響下,圖檔的辨識力退化
% 所作的權宜措施。可能是因為 TeXLive2007 hyperref 裏的
% 客製 graphicx / dvipdfmx 的設定檔不夠新
\ifx\myworkflow\mydvipdfmxflow
	\DeclareGraphicsExtensions{.pdf,.png,.jpg,.eps}
	\DeclareGraphicsRule{.pdf}{eps}{.bb}{}
	\DeclareGraphicsRule{.png}{eps}{.bb}{}
	\DeclareGraphicsRule{.jpg}{eps}{.bb}{}
\fi

% global CJK setting
\CJKindent  %%% ZZZ %%%  段首內縮兩格

% 載入中文名詞的定義:例如,Figure -->「圖」, Chapter -->「第 x 章」
%
% this file is encoded in utf-8
% v2.0 (Apr. 5, 2009)

% 下列中文名詞的定義,如果以註解方式關閉取消,
% 則會以系統原先的預設值 (英文) 替代
% 名詞 \prechaptername 預設值為 Chapter
% 名詞 \postchaptername 預設值為空字串
% 名詞 \tablename 預設值為 Table
% 名詞 \figurename 預設值為 Figure
\renewcommand\prechaptername{第} % 出現在每一章的開頭的「第 x 章」
\renewcommand\postchaptername{章}
\renewcommand{\tablename}{表} % 在文章中 table caption 會以「表 x」表示
\renewcommand{\figurename}{圖} % 在文章中 figure caption 會以「圖 x」表示

% 下列中文名詞的定義,用於論文固定的各部分之命名 (出現於目錄與該頁標題)
\newcommand{\nameInnerCover}{書名頁}
\newcommand{\nameCommitteeForm}{論文口試委員審定書}
\newcommand{\nameCopyrightForm}{授權書}
\newcommand{\nameCabstract}{中文摘要}
\newcommand{\nameEabstract}{英文摘要}
\newcommand{\nameAckn}{誌謝}
\newcommand{\nameToc}{目錄}
\newcommand{\nameLot}{表目錄}
\newcommand{\nameTof}{圖目錄}
\newcommand{\nameSlist}{符號說明}
\newcommand{\nameRef}{參考文獻}
\newcommand{\nameVita}{自傳}



% 如果不需要以中文數字一、二、三呈現章別,例如「第一章」
% 則請把下列介於 >>> 與 <<< 之間
% 的文字行關掉 (行首加上百分號), 會以「第 1 章」呈現
%% 中文數字章別 >>>
%
% this file is encoded in utf-8
% v2.0 (Apr. 5, 2009)

% 請依需要選擇其中一種表現方式,把它所對應的指令列打開,其他沒有用到的表現方式的對應指令列請關閉。(用行首百分號)

%% 第一種目錄格式:
%%	1  簡介 ............................ 1
%%
%%      章別 (chapter counter) 「1」前後沒有其他文字,
%%
%%      內文章標題是
%%		第 1 章	簡介
%%	\tocprechaptername, \tocpostchaptername 都設成沒有內容的空字串
%%	\tocChNumberWidth 設成 1.4em (預設)
%%      底下三行指令請打開
%\renewcommand\tocprechaptername{}
%\renewcommand\tocpostchaptername{}
%\setlength{\tocChNumberWidth}{1.4em}


%% 第二種目錄格式:
%%	一、簡介 ............................ 1
%%
%%      章別 (chapter counter) 「一」前沒有文字,後有頓號,
%%
%%      內文章標題是
%%		第一章		簡介
%%	\tocprechaptername 設成沒有內容的空字串
%%	\tocpostchaptername 設成頓號
%%	\tocChNumberWidth 設成 2em
%%      底下四行指令請打開 (預設)
\renewcommand\countermapping[1]{\CJKnumber{#1}}
\renewcommand\tocprechaptername{}
\renewcommand\tocpostchaptername{、}
\setlength{\tocChNumberWidth}{2em}


%% 第三種目錄格式:
%%	第一章、簡介 ......................... 1
%%
%%      章別 (chapter counter) 「一」前有「第」,後有「章」與頓號,
%%      內文章標題是
%%		第一章		簡介
%%	\tocprechaptername 設成「第」
%%	\tocpostchaptername 設成「章、」
%%	\tocChNumberWidth 設成 3em
%%      底下四行指令請打開
%\renewcommand\countermapping[1]{\CJKnumber{#1}}
%\renewcommand\tocprechaptername{第}
%\renewcommand\tocpostchaptername{章、}
%\setlength{\tocChNumberWidth}{3em}



%% 可以依照需要作彈性的設定
%%
%% 章別 (數字,包括後面的字串) 的寬度 \tocChNumberWidth,
%% 會影響章名與章別之間的間隔 (太少則相疊,太多則留白)
%% 建議設成 \tocpostchaptername 內容字數加一,做為 em 的倍數,
%% 但至少也要有 1.4 倍。

%% <<< 中文數字章別

%%% 以下是載入前頁、本文、後頁
% 請勿更動
% 如需針對個別章節獨立編譯
% 請在 my_chapters.tex 檔裡對個別章節的 \input 指令以行首百分號方式做開關。

\NTUtitlepage  % 產生論文封面

\newpage
\setcounter{page}{1}
\pagenumbering{roman}

\NTUoralpage  % 產生口試委員會審定書

\mydoublespacing
%\begin{acknowledgement} %誌謝
   %請在這裡寫您的誌謝辭 
%\end{acknowledgement}

\begin{zhAbstract}  %中文摘要
  中文摘要
\end{zhAbstract}

{
%\zhKaiFont
\mysinglespacing\selectfont
\tableofcontents %目錄

\listoffigures  %圖目錄

\listoftables  %表目錄
\par
}

\newpage
\setcounter{page}{1}
\pagenumbering{arabic}


\chapter{導論}
    Latex Rocks! \cite{T1}

%
% this file is encoded in utf-8
% v2.0 (Apr. 5, 2009)

%%% 參考文獻
\newpage
\phantomsection % for hyperref to register this
\addcontentsline{toc}{chapter}{\nameRef}
\renewcommand{\bibname}{\protect\makebox[5cm][s]{\nameRef}}
%  \makebox{} is fragile; need protect
\bibliographystyle{IEEEbib}  % 使用 IEEE Trans 期刊格式
\bibliography{template_bib}


%%% 附錄
%\input{my_appendix.tex}

%%% 自傳
%\newpage
%\chapter*{\protect\makebox[5cm][s]{\nameVita}} % \makebox{} is fragile; need protect
%\phantomsection % for hyperref to register this
%\addcontentsline{toc}{chapter}{\nameVita}
%\input{my_vita.tex}



\clearpage % to make sure all CJK characters are processed
\end{CJK}  %%% ZZZ %%%
\end{document} 
 
